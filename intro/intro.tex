

I've never been very good or interested in playing board games, but I've always had a fascination with how to play them well. When I was 13 years old I started programming, and one of my first projects was to write an AI for tic-tac-toe. This is rather easy, as the game is tiny, but it was a good project for teaching me to program. A few years later I was introduced to mancala, and as a way to understand the game better, I decided to write a program to play it, and in the process, independently reinvented minimax and rediscovered that many games are zero-sum. My mancala program never got to be any good as I didn't know anything about alpha-beta and wikipedia hadn't been invented yet, but I have always had a greater interest in understanding how the mechanics of the game work than actually playing the game. Writing a strong program is a great challenge, and a very satisfying one if your program becomes a stronger player than you are yourself.

In late 2008 I was introduced to Pentago, an interesting game invented in 2005. It is a 2-player game played on a 6x6 board where each turn is composed of placing a stone and rotating a 3x3 quadrant with the goal of forming 5 in a row. After playing a few rounds and losing badly, I decided to figure out how to write a program to play it so that I could better understand how to play. During a few rounds of play I devised a simple heuristic, which on its own is very weak, but when used with the common alpha-beta algorithm is quite strong. With some optimization, my program, Pentagod, became the strongest pentago computer program in the world to my knowledge, easily crushing me and my friends. In hindsight this program is only the best in the world because no one else has seriously tried, but it was a fun project either way.

In early 2010, while taking a computer science course in computer game AI, I was tasked with writing a program to play Havannah. Basing my program, Castro, on my earlier work on Pentagod, my program became reasonably strong by program standards, but still quite weak by human standards. In fact the creator of Havannah was so certain that programs would remain weak that he issued a challenge for \euro 1000 to anyone who can beat him in only one in ten games on size 10 by 2012. I continued working on my program after the course finished, implementing techniques mentioned in class or used in other games, trying to use the theoretical properties used in the related game Hex, optimizing my code for pure efficiency and parallelism, and coming up with Havannah specific techniques. In September 2010 I went to Kanazawa to compete in the Computer Games world championship and won 15 out of 16 games, winning the tournament. Soon after I attempted to solve size 4, a small version of the game, and succeeded in January 2011. 

This thesis is the story of what it takes to write a strong Havannah player, and how this player was used as the basis of solving size 4 Havannah. Chapter 2 explains the required background knowledge for the algorithms I use in the rest of thesis. Chapter 3 introduces several properties of the game itself that make writing a program challenging, and a few that can be exploited to increase the playing strength. Chapter 4 explains how the general techniques were adapted to Havannah and introduces a few Havannah specific heuristics which together lead to a tournament level program. Chapter 5 explains how the player was used to solve size 4 Havannah and the extra techniques needed to accomplish this goal. Finally it presents the solution to size 4 Havannah.


