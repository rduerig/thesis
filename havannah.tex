% vim: set wrap

\section{Rules of Havannah}


\begin{figure}
\centering
\begin{HavannahBoard}[board size=6,coordinate style=classical]
\HStoneGroup[color=black,label=$\mathcal F$]{e10,f10,g10,g9,h9,h8,i8,j8,h7,h6,h5,i5,i4,k8}
\HStoneGroup[color=white,label=$\mathcal B$]{a1,a2,b3,c3,d4,e4,e3,e2,f2,f1}
\HStoneGroup[color=gray,label=$\mathcal R$]{e7,e8,d8,c8,b7,b6,b5,c5,d6}
\end{HavannahBoard}
\caption[The Three Havannah Winning Conditions]{The Three Havannah Winning Conditions, as shown on a size 6 Havannah board}
\label{fig:rules}
\end{figure}

Havannah is a connection game invented in 1979 by Christian Freeling. It is a two player, zero-sum, perfect information game played on a hexagonal board. Each turn a player places a stone on the board in alternating play. Stones are never moved nor removed after their initial placement. A \textit{group} or \textit{chain} is a set of connected stones of the same colour. The game ends when one of the players completes one of the three winning conditions which are shown in Figure \ref{fig:rules}:
\begin{itemize}
	\setlength{\itemsep}{0pt}
	\setlength{\parskip}{0pt}
	\setlength{\parsep}{0pt}
	\item A \textit{Bridge} is a group of stones that connects any 2 corners, for example the stones labelled $\mathcal B$ in Figure \ref{fig:rules}.
	\item A \textit{Fork} is a group of stones that connects any 3 edges (corners are not part of edges), for example the stones labelled $\mathcal F$ in Figure \ref{fig:rules}.
	\item A \textit{Ring} is a group of stones that surround at least one cell (which can be empty or filled by either player), for example the stones labelled $\mathcal R$ in Figure \ref{fig:rules}.
\end{itemize}

The \textit{size} of the board is defined as the number of cells along one edge, so the board in Figure \ref{fig:rules} is size 6. Havannah can be played on any size board, but is usually played on boards ranging from size 4 to size 10. Stronger players prefer bigger boards, due to the larger component of strategy compared to the small boards where tactics dominate. In 2002, Christian Freeling offered \euro 1000 for any program that beats him in just one in ten games on size 10 by 2012.

Havannah is played by a few thousand players around the world, primarily on Little Golem\footnote{\url{http://littlegolem.net}} and similar sites. It is also played by computer programs at the International Computer Games Association annual Computer Olympiads.\footnote{\url{http://www.grappa.univ-lille3.fr/icga/}}



\section{Coordinate System}

Several coordinate systems for specifying board locations exist. The one that will be used here was chosen because it has some nice mathematical properties\footnote{\url{http://www.iwriteiam.nl/Havannah.html}} and because it is used in HavannahGui\footnote{\url{http://mgame99.mg.funpic.de/havannah.php}} and in the Little Golem\footnote{\url{http://www.littlegolem.net}} SGF files. An example board is shown in Figure \ref{fig:coordinatesboard} with each cell marked with its coordinate location. Figure \ref{fig:coordinatessquare} shows the same board as represented on a square grid. The empty points in the square grid are unused for the purposes of this representation. In the square representation connections are valid in the vertical, horizontal and $x=y$ directions, but not in the $x=-y$ direction. This square representation is often used to represent the board in memory. The size of the board is the number of cells along one short edge, or the radius of the board, not the diameter. Given this representation, the distance $d$ between any two points $(x_1,y_1)$ and $(x_2,y_2)$ can be calculated as: $$d = (|x_1-x_2| + |y_1-y_2| + |(x_1-y_1)-(x_2-y_2)|)/2$$


\begin{figure}
\centering
	\subfloat[]{\label{fig:coordinatesboard}
		\begin{HavannahBoard}[board size=3,coordinate style=classical,hex height=22pt]
		\HStoneGroup[color=white,label=a1]{a1}
		\HStoneGroup[color=white,label=a2]{a2}
		\HStoneGroup[color=white,label=a3]{a3}
		\HStoneGroup[color=white,label=b1]{b1}
		\HStoneGroup[color=white,label=b2]{b2}
		\HStoneGroup[color=white,label=b3]{b3}
		\HStoneGroup[color=white,label=b4]{b4}
		\HStoneGroup[color=white,label=c1]{c1}
		\HStoneGroup[color=white,label=c2]{c2}
		\HStoneGroup[color=white,label=c3]{c3}
		\HStoneGroup[color=white,label=c4]{c4}
		\HStoneGroup[color=white,label=c5]{c5}
		\HStoneGroup[color=white,label=d2]{d2}
		\HStoneGroup[color=white,label=d3]{d3}
		\HStoneGroup[color=white,label=d4]{d4}
		\HStoneGroup[color=white,label=d5]{d5}
		\HStoneGroup[color=white,label=e3]{e3}
		\HStoneGroup[color=white,label=e4]{e4}
		\HStoneGroup[color=white,label=e5]{e5}
		\end{HavannahBoard}
	}
	\subfloat[]{\label{fig:coordinatessquare}
		\raisebox{58pt}{
		\begin{tabular}{c|ccccc}
		  &  1 &  2 &  3 &  4 &  5 \\ \hline
		a & a1 & a2 & a3 &    &    \\
		b & b1 & b2 & b3 & b4 &    \\
		c & c1 & c2 & c3 & c4 & c5 \\
		d &    & d2 & d3 & d4 & d5 \\
		e &    &    & e3 & e4 & e5 \\
		\end{tabular}
		}
	}
\caption[The Coordinate System]{The Coordinate System (a) as drawn on a size 3 board. (b) as represented on a square grid.}
\label{fig:coordinates}
\end{figure}


\section{State Space}

A crude overestimation of the number of Havannah states is $T_0(n) = 3^n$ where $n$ is the number of cells on the board. This includes many states that are unreachable purely based on the players having an uneven number of moves, such as all cells being played by player 1. A more accurate estimate is the sum of states where both players have made equal number of moves plus the sum of all states where player 1 has made one more move. This can be expressed as the formula:
$$T_1(n) = \sum_{i = 0}^{(n-1)/2} {n \choose i}*{n-i \choose i} + {n \choose i + 1}*{n - i - 1 \choose i}$$
These do not take symmetry or rotations into account, which gives approximately a 12 fold reduction in states: $T_2(n) = T_1(n)/12$. None of these approximations take the rules of the game into account, so includes positions where both players have winning formations or one player has multiple winning formations, but this is a much harder condition to approximate.

The state space complexity of Havannah is shown in Table \ref{table:complexity}, with various other board games listed for comparison. The Hex state space size is calculated with the same formula except only two-fold symmetry. By comparison with the games that have previously been solved, size 3 Havannah should be trivial to solve, size 4 should be hard but possible with brute force, and size 5 may be possible but only if strong mathematical properties can be found to dramatically reduce the search space as is done in Hex.

\begin{table}
	\centering
	\begin{tabular}{lrr|lrc}
	Havannah  & Cells & States $(T_2)$      & Game       & States     & Ref \\ \hline
	*Size 3   &    19 & $2 \times 10^{7}$   & *Connect 4 & $10^{14}$  & \cite{trompconnect4} \\
	*Size 4   &    37 & $6 \times 10^{15}$  & *Checkers  & $10^{20}$  & \cite{schaeffer1998solving}\\
	Size 5    &    61 & $1 \times 10^{27}$  & *Hex 8x8   & $10^{29}$  & \cite{henderson2009solving} $T_1(8^2)/2$\\
	Size 6    &    91 & $2 \times 10^{41}$  & Go 9x9     & $10^{38}$  & \cite{tromp2007combinatorics}\\
	Size 7    &   127 & $3 \times 10^{58}$  & Chess      & $10^{46}$  & \cite{tromp2010chess}\\
	Size 8    &   169 & $3 \times 10^{78}$  & Hex 11x11  & $10^{56}$  & $T_1(11^2)/2$\\
	Size 9    &   217 & $2 \times 10^{101}$ & Go 13x13   & $10^{80}$  & \cite{tromp2007combinatorics}\\
	Size 10   &   271 & $1 \times 10^{127}$ & Go 19x19   & $10^{171}$ & \cite{tromp2007combinatorics}\\
	\end{tabular}
	\caption[State Space Complexity of Havannah]{State Space Complexity of Havannah. Other board games are shown for comparison, * means solved.}
	\label{table:complexity}
\end{table}

\section{Properties of Havannah}\label{sec:properties}

Havannah is considered hard for computers to play for several reasons, including the lack of a good heuristic evaluation function, few expert games, and a large state space complexity. Havannah is often compared to Hex, which is also a connection game, but few of the mathematical properties of Hex apply in Havannah. In this section some properties of Havannah are presented, especially in contrast to the better known properties of Hex.



\subsection{Virtual Connections}\label{sec:vc}

A virtual connection (VC) is a connection between two stones, or a stone and an edge, that can be completed even if the opponent makes the first move. A simple virtual connection is shown in Figure \ref{fig:simplevc}. The two black stones are virtually connected, because if white plays in one of the two marked cells, black can complete the connection by playing in the other. In Hex, virtual connections are guaranteed, since there is no reason to not complete the connection, but this is not true in Havannah. In Figure \ref{fig:breakvc}, black has a virtual connection between his two groups, but white can force a defence against a ring threat as shown in Figure \ref{fig:brokenvc}, which allows white to sever the black virtual connection. Such threats are rare in practice, but cannot be ignored.

Figure \ref{fig:falsevc} shows a state where black's three groups intuitively look to be virtually connected, but aren't. If white plays in the center, black can choose which of his two top groups to connect to the bottom group, but he can't connect all three, as shown in Figure \ref{fig:falsevcbroken}. If white didn't have the first cell, then black could place there in move four, connecting all three groups.

\begin{figure}
  \centering

	\subfloat[]{\label{fig:simplevc}
		\begin{HexBoard}[board size=2,show coordinates=false,show hexes=true]
		\HHexGroup {a1,b1,a2,b2}
		\draw [thick,dotted] (a1)..controls(a2)..(b2);
		\draw [thick,dotted] (a1)..controls(b1)..(b2);
		\HStoneGroup[color=black]{a1,b2}
		\end{HexBoard}
	}
	\subfloat[]{\label{fig:breakvc}
		\begin{HavannahBoard}[board size=3,coordinate style=classical,show coordinates=false]
		\HStoneGroup[color=black]{d2,d3,c4,b4}
		\HStoneGroup[color=white]{b1,c2,a2,b3}
		\draw [thick,dotted] (d3)..controls(c3)..(c4);
		\draw [thick,dotted] (d3)..controls(d4)..(c4);
		\end{HavannahBoard}
	}
	\subfloat[]{\label{fig:brokenvc}
		\begin{HavannahBoard}[board size=3,coordinate style=classical,show coordinates=false]
		\HStoneGroup[color=black]{d2,d3,c4,b4}
		\HStoneGroup[color=white]{b1,c2,a2,b3}
		\HGame{c3,a1,d4}
		\end{HavannahBoard}
	}
	\subfloat[]{\label{fig:falsevc}
		\begin{HavannahBoard}[board size=3,coordinate style=classical,show coordinates=false]
		\HStoneGroup[color=black]{d3,c4,b2, e3,c5,a1}
		\HStoneGroup[color=white]{d4}
		\end{HavannahBoard}
	}
	\subfloat[]{\label{fig:falsevcbroken}
		\begin{HavannahBoard}[board size=3,coordinate style=classical,show coordinates=false]
		\HStoneGroup[color=black]{d3,c4,b2, e3,c5,a1}
		\HStoneGroup[color=white]{d4}
		\HGame{c3,c2,b3}
		\end{HavannahBoard}
	}

	\caption[Virtual Connections]{(a) A simple virtual connection. (b) Virtual connections are not guaranteed. (c) A threat can force a response other than maintaining the connection. (d) A false virtual connection between 3 groups can be broken, as shown in (e)}
	\label{fig:ringvc}
\end{figure}


\subsection{Frame}

A \textit{frame} is a series of virtually connected stones or chains of stones that if connected would complete a winning condition. The length of the frame is the number of moves needed to complete the winning chain. Several frames are shown in Figure \ref{fig:race}.


\subsection{Simultaneous Forced Wins: Race to Win}

\begin{figure}
	\centering
	\subfloat[]{\label{fig:racea}
		\begin{HavannahBoard}[board size=4,coordinate style=classical,show coordinates=false]
		\HStoneGroup[color=white]{d1,e3,g4}
		\HStoneGroup[color=black]{a4,c5,d7}
		\end{HavannahBoard}
	}
	\subfloat[]{\label{fig:raceb}
		\begin{HavannahBoard}[board size=4,coordinate style=classical,show coordinates=false]
		\HStoneGroup[color=white]{b5,c5,c4,b3,c2}
		\HStoneGroup[color=black]{f6,f5,e3,f7}
		\end{HavannahBoard}
	}
	\subfloat[]{\label{fig:racec}
		\begin{HavannahBoard}[board size=4,coordinate style=classical,show coordinates=false]
		\HStoneGroup[color=white]{b5,c5,c4,b3,c2}
		\HStoneGroup[color=black]{f6,f5,e3,f7}
		\HGame[first player=black]{g6,a3,a4,b2,f4,c1}
		\end{HavannahBoard}
	}
	\caption[Simultaneous Forced Wins: Race to Win]{(a) Both players have a forced win in 2 moves. (b) Both players have a forced win in 3 moves. (c) Continuing from (b) with Black to move, White wins
by threatening a faster ring connection.}
	\label{fig:race}
\end{figure}


In Hex, winning formations are mutually exclusive, so if one player has a forced win through virtual connections, he is guaranteed to win. This is because the winning conditions are side-to-side VCs which cross, so if one player has a side-to-side VC, the other player cannot also have one.

In Havannah winning formations are not mutually exclusive. Figure \ref{fig:rules}, though not a valid board configuration, shows three completed winning formations at the same time. While virtual connections can be broken, the formations needed to do so are often not present, in which case the virtual connections are guaranteed. Figure \ref{fig:racea} shows a situation where both players have a frame of length 2. This means they can force a win in two moves, so the first player to make a move wins. Figure \ref{fig:raceb} shows a situation where both players have a forced win in three moves, with black to move. One of white's moves threatens a faster ring victory, and although it is easily blocked, it gives white the move advantage and the win, as shown in Figure \ref{fig:racec}.



\subsection{Dead Cells}

Strong Hex programs reduce the moves under consideration by avoiding playing in dead cells. Dead cells are cells that provably cannot affect the outcome of the game. They are dead because any chain of stones that passes through them already has a path through the existing stones. The five smallest Hex dead cell patterns are shown in Figure \ref{fig:hexdeadcells}. Hex playing programs use these patterns to detect dead cells, then never consider playing in those cells, thereby reducing the branching factor.

\begin{figure}
  \centering
\begin{tabular}{ccccc}
\begin{HavannahBoard}[board size=2,coordinate style=classical,show coordinates=false]
\HStoneGroup[color=white]{a1,b1,c2,c3}
\HStoneGroup[color=black dot]{b2}
\end{HavannahBoard}
&
\begin{HavannahBoard}[board size=2,coordinate style=classical,show coordinates=false]
\HStoneGroup[color=black]{b3}
\HStoneGroup[color=white]{a1,b1,c2}
\HStoneGroup[color=black dot]{b2}
\end{HavannahBoard}
&
\begin{HavannahBoard}[board size=2,coordinate style=classical,show coordinates=false]
\HStoneGroup[color=black]{a2,b3}
\HStoneGroup[color=white]{b1,c2}
\HStoneGroup[color=black dot]{b2}
\end{HavannahBoard}
&
\begin{HavannahBoard}[board size=2,coordinate style=classical,show coordinates=false]
\HStoneGroup[color=black]{a1,a2,b3}
\HStoneGroup[color=white]{c2}
\HStoneGroup[color=black dot]{b2}
\end{HavannahBoard}
&
\begin{HavannahBoard}[board size=2,coordinate style=classical,show coordinates=false]
\HStoneGroup[color=black]{a1,a2,b3,c3}
\HStoneGroup[color=black dot]{b2}
\end{HavannahBoard}

\end{tabular}
	\caption[Hex-dead Cell Patterns]{Hex-dead Cell Patterns. The cell marked with a dot cannot help either player form a winning connection in Hex.}
	\label{fig:hexdeadcells}
\end{figure}

Unfortunately the Hex-dead cell patterns don't apply in Havannah, because the Hex-dead cells can have an effect on the outcome of the game in Havannah. While they cannot affect a fork or a bridge win, they can still be part of a winning ring which surrounds one of the existing stones as illustrated in Figure \ref{fig:ringdeadcells}. This idea can be used against all five of the smallest Hex-dead cell patterns in numerous ways.


\begin{figure}
  \centering
\begin{tabular}{ccc}
\begin{HavannahBoard}[board size=3,coordinate style=classical,show coordinates=false]
\HStoneGroup[color=white]{b2,c2,d3,d4}
\HStoneGroup[color=black dot]{c3}
\end{HavannahBoard}
&
\begin{HavannahBoard}[board size=3,coordinate style=classical,show coordinates=false]
\HStoneGroup[color=white]{b2,c2,d3,d4}
\HStoneGroup[color=white,label=1]{b1}
\HStoneGroup[color=white,label=2]{c1}
\HStoneGroup[color=white,label=3]{d2}
\HStoneGroup[color=white,label=4]{c3}
\end{HavannahBoard}
&
\begin{HavannahBoard}[board size=3,coordinate style=classical,show coordinates=false]
\HStoneGroup[color=white]{b1,b2,c1,c3,d2,d3}
\HStoneGroup[color=white]{c2,d4}
\draw [thick]    (b1) -- (c1) -- (d2) -- (d3) -- (c3) -- (b2) -- (b1);
\end{HavannahBoard}
\end{tabular}
	\caption[Hex-dead Cells are not Havannah-dead Cells]{Hex-dead Cells are not Havannah-dead Cells. Starting with the Hex-dead cell pattern on the left, and adding stones 1-4 leads to the ring on the right which intersects the Hex-dead cell}
	\label{fig:ringdeadcells}
\end{figure}

Note that if the ring is made bigger than a simple 6-ring, any ring that uses the dead cell would also work going around it. To create Havannah dead cell patterns, we add a stone of the opposing colour next to each existing stone to block the encircling 6-ring. A few examples are shown in Figure \ref{fig:havdeadcells}. Unfortunately these patterns are so rare that they aren't worth looking for explicitly.

\begin{figure}
  \centering
\begin{tabular}{ccccc}

\begin{HavannahBoard}[board size=3,coordinate style=classical,show coordinates=false,hex height=14pt]
\HStoneGroup[color=white]{b2,c2,d3,d4}
\HStoneGroup[color=black]{b1,e4}
\HStoneGroup[color=black dot]{c3}
\end{HavannahBoard}
&
\begin{HavannahBoard}[board size=3,coordinate style=classical,show coordinates=false,hex height=14pt]
\HStoneGroup[color=white]{b2,c2,d3, b4}
\HStoneGroup[color=black]{c4, b1,e3}
\HStoneGroup[color=black dot]{c3}
\end{HavannahBoard}
&
\begin{HavannahBoard}[board size=3,coordinate style=classical,show coordinates=false,hex height=14pt]
\HStoneGroup[color=white]{c2,d3, b4}
\HStoneGroup[color=black]{b3,c4, d2}
\HStoneGroup[color=black dot]{c3}
\end{HavannahBoard}
&
\begin{HavannahBoard}[board size=3,coordinate style=classical,show coordinates=false,hex height=14pt]
\HStoneGroup[color=white]{d3, a2,b4}
\HStoneGroup[color=black]{b2,b3,c4, e3}
\HStoneGroup[color=black dot]{c3}
\end{HavannahBoard}
&
\begin{HavannahBoard}[board size=3,coordinate style=classical,show coordinates=false,hex height=14pt]
\HStoneGroup[color=black]{b2,b3,c4,d4}
\HStoneGroup[color=white]{b1,a3,d5}
\HStoneGroup[color=black dot]{c3}
\end{HavannahBoard}

\end{tabular}
	\caption[Havannah Dead Cell Patterns]{Some Havannah Dead Cell Patterns. The cells marked with the dot are dead in Havannah.}
	\label{fig:havdeadcells}
\end{figure}





\subsection{Draws}

Unlike Hex, where a filled board must have a winner, draws are possible in Havannah. Figure \ref{fig:drawfilled} shows a filled board of a game that ended in a draw. Backing up a few moves we see in Figure \ref{fig:drawnowin} that after move 30 no wins are possible even if one of the players were to pass all their remaining moves. The game can be proven as a draw at least as early as move 20, shown in Figure \ref{fig:drawproven}. Draws are possible on all board sizes above 3. They occur occasionally on size 4, but are rare on size 5 and above. Section \ref{sec:drawdetect} will describe a technique for detecting draws once no wins are possible.

\begin{figure}
	\centering
	\subfloat[]{\label{fig:drawfilled}
		\begin{HavannahBoard}[board size=4,coordinate style=classical,show coordinates=false]
		\HGame{g7,a1,f5,g4,e3,d2,e2,d1,c2,d3,c3,f7,d6,b4,a4,b5,a3,e5,c4,b3, a2,c1,b1,f6,e4,d4,g6,e7,d7,b2,c5, f3,f4,g5,d5,e6,c6}
		%playgame g7 a1 f5 g4 e3 d2 e2 d1 c2 d3 c3 f7 d6 b4 a4 b5 a3 e5 c4 b3 a2 c1 b1 f6 e4 d4 g6 e7 d7 b2 c5 f3 f4 g5 d5 e6 c6
		\end{HavannahBoard}
	}
	\subfloat[]{\label{fig:drawnowin}
		\begin{HavannahBoard}[board size=4,coordinate style=classical,show coordinates=false]
		\HGame{g7,a1,f5,g4,e3,d2,e2,d1,c2,d3,c3,f7,d6,b4,a4,b5,a3,e5,c4,b3, a2,c1,b1,f6,e4,d4,g6,e7,d7,b2,c5}
		\end{HavannahBoard}
	}
	\subfloat[]{\label{fig:drawproven}
		\begin{HavannahBoard}[board size=4,coordinate style=classical,show coordinates=false]
		\HGame{g7,a1,f5,g4,e3,d2,e2,d1,c2,d3,c3,f7,d6,b4,a4,b5,a3,e5,c4,b3}%, a2,c1,b1,f6,e4,d4,g6,e7,d7,b2,c5}
		\end{HavannahBoard}
	}
	\caption[Havannah Draws]{(a) Filled board ending as a draw. (b) After move 30 no wins are possible. (c) Proven draw after move 20.}
	\label{fig:draw}
\end{figure}


\section{Summary}

Havannah is a recent game with fairly simple rules. It is played on a hex board, with the goal of connecting 2 corners or 3 edges, or forming a ring. It is generally played on board sizes 4 through 10. It is related to Hex, and shares many of the mathematical properties, but they are not as solid as in Hex. Virtual connections are formed the same way as in Hex, and are very important to Havannah strategy, but unlike Hex, they can be broken due to threats. Chains of virtual connections that, if completed, would form a win, are called frames. Multiple frames can co-exist on the board, and the player with the shortest frame will win unless an even shorter frame is constructed. Dead cells are cells can not affect the outcome of the game due to the pattern of stones that surround it, but the patterns are rare and more complex than the hex-dead cell patterns. Draws are rare but possible in Havannah, occurring when the board fills up without either player forming a winning formation.

