% This template was created May 4th, 2010 by Colin Ophus.  Since my thesis was accepted by the Library of Canada, I believe all of the formatting, page ordering, etc is correct.  To start, fill out all of the definitions below, ie every term after a \def. Then modify the existing chapters and create your own.

% This is a thesis template designed to meet the University of Alberta electronic submission standards.
% It comes with absolutely no guarantees that it will work at all.
% That being said, it should work first try with no need to install additional packages or style files. I designed it with PDFLaTeX in mind, so you should use .png and if you must .jpg and .tiff image files. For the more advanced users, PDF files can be used for any figure or if you change to a DVI intermediary you can use .eps format.

% Known bugs:
% - Unfortunately you will get ~9 warnings for the intro pages.  This is a consequence of removing the numbering from those pages (as required by the U of A faculty of grad studies and research.  I do not know how to correct this.
% - The preface page is double spaced in addition to the abstract page. This is not really a problem, but I would prefer to double space only the abstract.  I have not figured out a good way to do so though.

\documentclass[12pt, letterpaper]{report}

% Define global variables, like names and thesis title
% If you thesis title is more than three lines, you will need to adjust the spacing on the title page.
\def\name{Timo Ewalds}
\def\thesistitle{Playing and Solving Havannah using Monte Carlo Tree Search and Proof Number Search}
\def\supervisor{Jonathan Schaeffer, Ryan Hayward}
\def\coma{Prof Name}
\def\comb{Prof Name}
\def\comc{Prof Name}
\def\comd{Prof Name}
\def\come{Prof Name - External}
\def\superloc{Computing Science}
\def\loca{Department}
\def\locb{Department}
\def\locc{Department}
\def\locd{Department}
\def\loce{Department, Institution}
\def\program{Masters}  % Your degree
\def\school{University of Alberta}
\def\semester{Fall 2011}  % The convocation period when you submit your thesis:  Spring or Winter, then year
\def\dept{Computing Science}  % Your department

% Add command for degree symbol:   \degree
\newcommand{\degree}{\ensuremath{^\circ}}

% Various packages and appearance changes
% Adjust if desired
\usepackage{amsmath, amssymb, amsthm}
\usepackage{graphicx,color}
\usepackage[left=1.5in, right=1.5in, top=1.5in, bottom=1in, includefoot, headheight=.5in]{geometry}
\parindent 0pt
\parskip 10pt
\renewcommand{\baselinestretch}{1.33}
\numberwithin{equation}{section}
\renewcommand{\bibname}{References}
\renewcommand{\contentsname}{Contents}
\pagenumbering{roman}

\usepackage{havannah}
\renewcommand\HDrawHex{\draw[fill=gray!35]}


% Bibliography stuff
\usepackage[square, comma, numbers, sort&compress]{natbib}
\renewcommand{\bibsep}{10pt}
\bibliographystyle{unsrtnat}

% Customising headers 
\usepackage{fancyhdr}
\pagestyle{fancy}
\rhead{}
\lhead{\nouppercase{\textsc{\leftmark}}}
\renewcommand{\headrulewidth}{0pt}
\makeatletter
\renewcommand{\chaptermark}[1]{\markboth{\textsc{\@chapapp}\ \thechapter:\ #1}{}}
\makeatother%

% New chapter headings
\usepackage[grey,utopia]{quotchap}

% PDF hyperlinks
% This is where you change their colouring - feel free to go wild or conservative, it is your thesis after all
\usepackage[colorlinks]{hyperref}
\usepackage[figure,table]{hypcap}
\hypersetup{
  bookmarksnumbered,
  pdfstartview={FitH},
  citecolor={red},
  linkcolor={red},
  urlcolor={red},
  pdfpagemode={UseOutlines}
}
\makeatletter
\newcommand\org@hypertarget{}
\let\org@hypertarget\hypertarget
\renewcommand\hypertarget[2]{%
  \Hy@raisedlink{\org@hypertarget{#1}{}}#2%
} 
\makeatother

% Change bulleted lists - feel free to modify this
\renewcommand{\labelitemi}{$\blacktriangleright$}





% Actual text of the thesis starts here
\begin{document}
  \pagenumbering{roman}
  \setcounter{page}{-99}  % So that page numbering won't interfere, -ve numbers don't show
  \thispagestyle{empty}
  

  % Titlepage
  \pdfbookmark[0]{Prefatory Pages}{prefatory}
  \pdfbookmark[1]{Title}{title}
%  \vspace{.25in}
%  \title{
  \begin{center}
    \Large{\textbf{\school}}  \\ [.6in]
    \Large{\textbf{\thesistitle}} \\ [.1in]
    \normalsize{by} \\ [.1in]
    \Large{\textbf{\name}}  \\ [.6in]
    \normalsize{A thesis submitted to the Faculty of Graduate Studies and Research \\ 
    in partial fulfillment of the requirements for the degree of} \\ [0.1in]
    \Large{\textbf{\program}} \\ [.1in]
    \normalsize{\dept} \\ [0.6in]  
    \scriptsize{\copyright\:\name} \\
    \scriptsize{\semester} \\
    \scriptsize{Edmonton, Alberta} \\ [0.6in]  
    % DO NOT modify this text, it is a requirement
    \scriptsize{Permission is hereby granted to the University of Alberta Libraries to reproduce single copies of this thesis and to lend or sell such copies for private, scholarly or scientific research purposes only. Where the thesis is converted to, or otherwise made available in digital form, the University of Alberta will advise potential users of the thesis of these terms.
    
The author reserves all other publication and other rights in association with the copyright in the thesis and, except as herein before provided, neither the thesis nor any substantial portion thereof may be printed or otherwise reproduced in any material form whatsoever without the author's prior written permission.}
  \end{center}

  % Examining committee page
  \newpage
  % Begin numbering the pages with Roman numerals
  \pdfbookmark[1]{Examining Committee}{examining}
  \chapter*{Examining Committee}
  \thispagestyle{empty}
     \supervisor, \; \superloc \\ \\
     \coma, \; \loca \\ \\
     \comb, \; \locb \\ \\
     \comc, \; \locc \\ \\
     \comd, \; \locd \\ \\
     \come, \; \loce
  
  
  % Dedication
   \newpage 
  \pdfbookmark[1]{Dedication}{dedication}
  \chapter*{}
  \thispagestyle{empty}
  This thesis is dedicated to\\
  A person or persons or perhaps some abstract concept.

  % Abstract
  \newpage 
  \pdfbookmark[1]{Abstract}{abstract}
  \chapter*{Abstract}
  \thispagestyle{empty}
  \vspace*{-0.7in}
  \renewcommand{\baselinestretch}{1.8}
  \normalsize{
  This is the abstract text. It MUST be double spaced. Don't forget that. Duis non sapien quis justo sagittis tempor id id odio. In varius porta sollicitudin. Lorem ipsum dolor sit amet, consectetur adipiscing elit. Suspendisse sodales sagittis ante, id ultrices ipsum sollicitudin quis. Ut fermentum ornare neque eu sollicitudin. Quisque neque massa, facilisis id congue ac, blandit vitae nulla. Donec vulputate scelerisque lorem quis tincidunt.
  } \vspace*{-0.2in} \\
  
  % A second paragraph if you need it
  \normalsize{
  This is a second abstract paragraph. if you need it. Nullam rutrum elit in magna porta vehicula id non elit. Nam varius ultricies lectus ac consequat. Duis tellus ligula, convallis ac pretium et, dignissim at enim. Cras turpis lorem, eleifend at imperdiet et, sollicitudin id ante. Nam blandit volutpat nisl, nec congue mi feugiat sed. Nunc accumsan, urna a elementum viverra, elit purus iaculis urna, vel vehicula est risus quis turpis. Ut laoreet scelerisque elit, a rhoncus nibh placerat in. Nulla feugiat ullamcorper justo quis adipiscing. Etiam dolor arcu, porta et dictum vitae, dictum eget dui.
  }
  \renewcommand{\baselinestretch}{1.33}
  
  % Preface
  \newpage 
  \pdfbookmark[1]{Preface}{preface}
  \chapter*{Preface}
  \thispagestyle{empty}
  \vspace*{-0.7in}
  Here we describe the layout of the thesis.  You should also include a one or two sentence description of each chapter, as such: Chapter \ref{intro} introduces the concepts used in this thesis. Chapter \ref{achapterlabel} describes a topic. Chapter \ref{bchapterlabel} talks about another one. Finally, chapter \ref{conc} wraps it all up. Note the hyperlinked references!
  
  % Also you can define a label anywhere using \label{blah}.  Then you can make links to it with \ref{blah}.  Very handy.

  % Acknowledgements
  \newpage 
  \pdfbookmark[1]{Acknowledgements}{acknowledgements}
   \chapter*{Acknowledgements}
   \thispagestyle{empty}
   \vspace*{-0.7in}
   \small{
  I thank so-and-so, as well as whatshisname. They were awesome.
  }
  
  % Table of contents
  \newpage
  \pdfbookmark[0]{Contents}{contents}
  \pdfbookmark[1]{Table of Contents}{toc}
  \normalsize
  \tableofcontents 
  
  % List of figures
  \newpage
  \pdfbookmark[1]{List of Figures}{listfigs}
  \listoffigures

  % List of tables
  \newpage  
  \pdfbookmark[1]{List of Tables}{listtables}
  \listoftables
  
  % List of symbols
  % I have included my own symbols as an example. Modify as you see fit.
  \newpage
  \thispagestyle{empty}
  \pdfbookmark[1]{List of Symbols}{listsymbols}
  \section*{}
  \begin{flushright}
    \huge{List of Symbols}
  \end{flushright}
  \vspace{0.4in}
  \begin{center}
    \begin{tabular}{rl}
      Symbol & Meaning\\
      \hline
%      $\alpha$      & Upper Bound \\
%      $\beta$        & Lower Bound \\
      $\delta$       & Proof number at OR node, disproof number at AND node \\
      $\phi$         & Disproof number at OR node, proof number at AND node \\
    \end{tabular}
  \end{center}
  
  % List of abbreviations
  % Again here are some examples
  \newpage
  \thispagestyle{empty}
  \pdfbookmark[1]{List of Abbreviations}{listabbrev}
  \section*{}
  \begin{flushright}
    \huge{List of Abbreviations}
  \end{flushright}
  \vspace{0.4in}
  \begin{center}
    \begin{tabular}{rl}
      Abbreviation & Meaning \\
      \hline
      $\alpha\beta$ & Alpha-Beta algorithm \\
      DFPN    & Depth First Proof Number search \\
      MCTS    & Monte Carlo Tree Search \\
      PNS      & Proof Number Search \\
      RAVE    & Rapid Action Value ... \\
      UCB      & Upper Confidence Bounds \\
      UCT      & Upper Confidence bounds as applied to Trees \\

    \end{tabular}
  \end{center}  
  
  
  

% ********************************************************************************
  % Main thesis chapters
  % Comment out whatever you are not working on if you want the thesis to buld faster
  \newpage
  % Begin numbering the pages with arabic numerals (main thesis body)
  \setcounter{page}{1}
  \pagenumbering{arabic}

  % Intro chapter, first numbered chapter
  \chapter[Introduction]{\label{intro} \LARGE Introduction, Motivation and Contributions}
  

People play games. People write programs to play games. Strong programs need many techniques. I apply these techniques to Havannah and come up with some new techniques, some Havannah specific, and some general as well as solve the smaller versions of Havannah.



  % Intro chapter, first numbered chapter
  \chapter[Background]{\label{intro} \LARGE General Game Playing Techniques}
  

Game playing programs all build a game tree, and then chose the most promising move at the root of the tree.

\section{Minimax}

The minimax algorithm is the foundation of all game playing algorithms and was invented before computers. The goal is the find the minimax value of a state or set of states, or equivalently for a set of moves, and then choose the move with the highest value. All values are from the perspective of the root player. The value of a node for the root player is the maximum of its children nodes, and the minimum for the opponents children. The pseudocode for a simple depth first search version is shown in Figure \ref{fig:minimaxcode}.

\begin{figure}

\begin{lstlisting}
int minimax(State state){
	if(state.terminal())
		return state.value();
	int value;
	if(state.player() == 1){
		value = -INF;
		foreach(state.successors as succ)
			value = max(value, minimax(succ));
	}else{
		value = INF;
		foreach(state.successors as succ)
			value = min(value, minimax(succ));
	}
	return value;
}
\end{lstlisting}

\caption{Minimax Pseudocode}
\label{fig:minimaxcode}

\end{figure}


\subsection{Negamax}

Minimax uses values as taken from a fixed perspective of the root player. This complicates the code with having to minimize for one player and maximize for the other. Noting that $max(a,b) = -min(-a,-b)$, the duplication can be removed by negating the value each time we switch perspective. In this setup all values returned from an evaluation function must be from the perspective of the player who is making the move. The pseudocode for this transformation is shown in Figure \ref{fig:negamaxcode}.

\begin{figure}

\begin{lstlisting}
int negamax(State state){
	if(state.terminal())
		return state.value();
	int value = -INF;
	foreach(state.successors as succ)
		value = max(value, -negamax(succ));
	return value;
}
\end{lstlisting}

\caption{Negamax Pseudocode}
\label{fig:negamaxcode}
\end{figure}

Several algorithms shown later reference the negamax formulation, and mean that the perspective shifts after every move.


\section{Alpha-Beta}\label{sec:alphabeta}

Alpha-beta ($\alpha\beta$) is a refinement of minimax, ignoring or pruning parts of the game tree that are provably unreachable if both players play perfectly. It maintains two bounds to store the minimum value each player is guaranteed given the tree searched so far. When these bounds meet or cross, this is called a cut-off, and the remaining moves don't need to be considered. 

The pseudocode for alpha-beta, written in the negamax formulation is shown in Figure \ref{fig:abcode}. It is a depth first implementation that returns after a maximum depth is reached. If a terminal node is found, the true value is returned, otherwise a heuristic value is returned.

\begin{figure}

\begin{lstlisting}
int alphabeta(State state, int depth, int alpha, int beta){
	if(state.terminal() || depth == 0)
		return state.value();
	foreach(state.successors as succ){
		alpha = max(alpha, -alphabeta(succ, depth-1, -beta, -alpha));
		if(alpha >= beta)
			break;
	}
	return alpha;
}
\end{lstlisting}

\caption{Alpha-beta Pseudocode, shown in the negamax formulation}
\label{fig:abcode}
\end{figure}

The runtime of alpha-beta is highly dependent on the branching factor $b$, search depth $d$, and the number of cut-offs. Minimax has a runtime of $b^d$, as does alpha-beta if it has no cut-offs. Given perfect move ordering, only the first move for the root player will need to be considered, leading to a runtime of $b^{d/2}$, or an exponential speedup. In general, we don't have perfect move ordering, so the runtime will be between these two extremes.

\subsection{Transposition Table}

Transpositions can lead to an exponential blowup in the search space. To minimize the number of transpositions reevaluated, all strong alpha-beta based programs use a transposition table. Transpositions are found by comparing hash values and indexing into a large table. Sometimes a hash table is used, but usually the number of nodes searched is too big to store in memory, so a simple replacement policy is used. The simplest is to use the hash value as an index into a large array of values, replacing the previous node that indexed to the same location.

In many games this leads to a large speedup as the number of nodes searched is decreased dramatically.


\subsection{Iterative Deepening}

The runtime of alpha-beta is exponential in the search depth, and the strength of a player is dependent on a large search depth. If the algorithm is stopped before completion, the best move may not have been explored at all, so a shallower search that finishes is likely better than a deeper search that doesn't. Thus we start with a shallow search, and run a deeper search if we have time. This is not a big waste of work since the majority of the runtime is spent at the deepest level anyway. Iterative deepening allows alpha-beta to act as a breadth-first search with the memory overhead of a depth-first search.

Iterative deepening, when combined with a transposition table, also gives better move ordering. A nodes value from the previous iteration is going to be a more accurate estimate of the value of a node than a heuristic estimate without a search. As we saw in section \ref{sec:alphabeta}, better move ordering can lead to an exponential speedup, easily offsetting the overhead from searching the shallow depths multiple times.

\subsection{History Heuristic}

A good move ordering can lead to many cutoffs and the associated speed increase. The history heuristic is a game independent move ordering heuristic that works by giving higher priority to moves that lead to cutoffs elsewhere in the tree. If a particular move gives a cutoff, it's quite likely that it will also give a cutoff for all of its siblings and so should have a higher priority there. This does assume that similar moves in different part of the tree are related.

\subsection{Other extensions}

Negascout, killer move, quiescence search, etc. Are these worth including at all?





\section{Proof Number Search} \label{sec:PNS}

pseudocode?

Proof Number Search (PNS) is a best-first search used to answer binary questions such as the outcome of a 2-player game starting from a given state. Being a binary outcome with the minimax property, it is well represented as an AND/OR tree when all values are from the perspective of the root player. Each node in the tree can have one of three values: Proven/Win, Disproven/Loss, or Unknown. All nodes store 2 numbers that show how close it is to be proven or disproven. The proof number (pn) is the minimum number of leaf nodes in the subtree that must be proven for the node to be proven. The disproof number (dn) is the minimum number of leaf nodes in the subtree that must be disproven for the node to be disproven. Some leaf nodes, if solved, will change the proof number of the root. Other leaf nodes, if solved, will change the disproof number of the root. Others, if solved, won't affect the proof or disproof numbers of the root. The Most Proving Nodes (MPN) are the intersection of the set that affect the proof number and the set that affect the disproof number at the root. Solving a Most Proving Node will definitely affect either the proof or disproof number of the root. Every tree is guaranteed to have at least one most proving node. Proof Number search grows its tree by continually expanding a most proving node. Proof Number search can be split into 3 phases: descent, expansion, and update.

The most proving node is found during the descent phase. It can be found by selecting the child with the minimum proof number when at an OR node and by selecting the child with the minimum disproof number when at an AND node, applying this iteratively until a leaf node is reached. This leaf node is an MPN.

Once the most proving node $n$ is found, it is expanded, initializing all non-terminal children with $n_i.pn = 1, n_i.dn = 1$, winning children with $n_i.pn = 0, n_i.dn = \infty$ and losing children with $n_i.pn = \infty, n_i.dn = 0 $, where $n_i$ refers to the $i^{th}$ child of $n$.

After expansion, the proof and disproof numbers of all the ancestors of the most proving node must be updated using these formulas. For OR nodes: $$ n.pn = \displaystyle\min\limits_{i=0}^k n_i.pn,\quad n.dn = \displaystyle\sum\limits_{i=0}^k n_i.dn $$ For AND nodes: $$ n.pn = \displaystyle\sum\limits_{i=0}^k n_i.pn, \quad n.dn = \displaystyle\min\limits_{i=0}^k n_i.dn $$ Note how this backs up a single win at an OR node as a win, or a single loss at an AND node as a loss. It also backs up all losses at an OR node as a loss, or all wins at an AND node as a win. % Everywhere else it represents the optimistic lower bound on the number of nodes that must be (dis)proven to (dis)prove the current node.

These three phases are repeated until the root is solved or the tree grows too big to be stored in memory. At the root, if $r.pn = 0$ it is solved as a win, or if $r.dn = 0$ it is solved as a loss, otherwise it is still unknown.

\begin{figure}
\centering
\ovalbox{
\begin{tikzpicture}[
	level distance=15mm,
	level 1/.style={sibling distance=60mm},
	level 2/.style={sibling distance=35mm},
	level 3/.style={sibling distance=15mm},
	]
\node [rectangle,draw] (z){$a \frac{1}{2}$}
  child {node [circle,draw] {$b \frac{1}{2}$}
    child {node [rectangle,draw] {$d \frac{0}{\infty}$}
      child {node [circle,draw,label=below:?] {$h \frac{1}{1}$}}
      child {node [circle,draw,label=below:win] {$i \frac{0}{\infty}$}}
    }
    child {node [rectangle,draw] {$e \frac{1}{2}$}
      child {node [circle,draw,label=below:?] {$j \frac{1}{1}$}}
      child {node [circle,draw,label=below:loss] {$k \frac{\infty}{0}$}}
      child {node [circle,draw,label=below:?] {$l \frac{1}{1}$}}
    }
  }
  child {node [circle,draw] {$c \frac{\infty}{0}$}
    child {node [rectangle,draw,label=below:loss] {$f \frac{\infty}{0}$}}
    child {node [rectangle,draw,label=below:?] {$g \frac{1}{1}$}}
  };
\end{tikzpicture}}
\caption{Proof Number search tree, Squares are OR nodes, Circles are AND nodes, proof numbers are on top, disproof numbers on the bottom, based on \cite{winands2003-PDS-PN}}
\label{fig:pntree}
\end{figure}

Consider the tree in Figure \ref{fig:pntree}. The most proving node is found by following the edges $a \rightarrow b \rightarrow e \rightarrow j$. If $j$ has a child that is a win, it would be backed up as a win at $j$, leading to a win at $e$, and a win at $b$, giving the root player a winning move from the root. With $a.pn = 1$ at the root, only 1 node was needed to be proven as a win for the root to also be proven as a win. If both $j$ and $l$ were proven to be losses, then $e$ would be a loss, leading $b$ to also be a loss, and consequently the root to also be a loss. This is reflected in $a.dn = 2$ at the root. If, however, $j$ has 1 non-terminal child $m$ and no terminal children, $m$ would have $m.pn = 1, m.dn = 1$ and would be the new MPN. If $j$ has 2 non-terminal children and no terminal children, $j.pn = 2, j.dn = 1$, and $l$ would be the new MPN. 

This algorithm selects nodes based on the shape and value of the tree, using no domain or game specific heuristic. It is guided towards slim parts of the tree, areas where there are few moves available, or where many moves are forced. In many games it is advantageous to have more moves available, or higher mobility, than your opponent. Proof Number search is very fast at solving these positions. In games or positions where the branching factor is constant or consistent, with few forced moves, Proof Number search approximates a slow breadth-first search, and thus isn't very fast.

Being a best-first search algorithm, the whole tree must be kept in memory, since any node could become the MPN and therefore be searched at any time. This makes it a very memory intensive search algorithm, with many of the variants attempting to reduce memory usage, allowing bigger problems to be solved.

One simple optimization is to stop the update phase once the proof and disproof numbers don't change. This often happens when siblings have the same value, causing the sibling to be the new MPN. A new search can be started from this node instead of from the root. A simple memory optimization is to remove and reuse the memory of subtrees under a proven or disproven node.

\subsection{The Negamax Formulation} \label{sec:NegaPDS}

\begin{figure}
\centering
\ovalbox{
\begin{tikzpicture}[
	level distance=15mm,
	level 1/.style={sibling distance=60mm},
	level 2/.style={sibling distance=35mm},
	level 3/.style={sibling distance=15mm},
	]
\node [rectangle,draw] (z){$a \frac{1}{2}$}
  child {node [rectangle,draw] {$b \frac{2}{1}$}
    child {node [rectangle,draw] {$d \frac{0}{\infty}$}
      child {node [rectangle,draw,label=below:?] {$h \frac{1}{1}$}}
      child {node [rectangle,draw,label=below:loss] {$i \frac{\infty}{0}$}}
    }
    child {node [rectangle,draw] {$e \frac{1}{2}$}
      child {node [rectangle,draw,label=below:?] {$j \frac{1}{1}$}}
      child {node [rectangle,draw,label=below:win] {$k \frac{0}{\infty}$}}
      child {node [rectangle,draw,label=below:?] {$l \frac{1}{1}$}}
    }
  }
  child {node [rectangle,draw] {$c \frac{0}{\infty}$}
    child {node [rectangle,draw,label=below:loss] {$f \frac{\infty}{0}$}}
    child {node [rectangle,draw,label=below:?] {$g \frac{1}{1}$}}
  };
\end{tikzpicture}}
\caption{Proof Number search tree using the Negamax formulation, all nodes are OR nodes, $\phi$ is on top, $\delta$ is below, based on \cite{winands2003-PDS-PN}}
\label{fig:negamaxtree}
\end{figure}

Just like minimax can be written in the negamax formulation, so too can proof number search. The Proof number at an OR node is the same as the Disproof number at an AND node, and is named $\phi$ (phi). Similarly, the Proof number at an AND node is the same as the Disproof number at an OR node, and is named $\delta$ (delta). Instead of considering all nodes to be from the one player's perspective, all nodes are considered to be from the player who is making the move at that node. This shift in perspective greatly simplifies the code for all variants of proof number search.

Figure \ref{fig:negamaxtree} shows the same tree as in Figure \ref{fig:pntree}, except using the negamax formulation. Note how all nodes are now OR nodes, and the proof and disproof numbers are exchanged in the nodes that were previously AND nodes.

Given this shift in perspective, the descent and update formulas need to be corrected. The new descent move selection is always to choose the child with the minimum delta. The new update formulas are: $$ n.\phi = \displaystyle\min\limits_{i=0}^k n_i.\delta, \quad n.\delta = \displaystyle\sum\limits_{i=0}^k n_i.\phi $$

\begin{figure}

\begin{lstlisting}
int pns(State state){
	Node root = initnode(state);
	while(root.phi != 0 && root.delta != 0) search(root, state);
	return (root.phi == 0 ? PROVEN : DISPROVEN);
}
void search(Node node, State state){
	if(node.numchildren == 0){ //found MPN
		foreach(state.successors as succ)
			node.addchild(initnode(succ));
	}else{
		do{
			Node child = node.child_min_delta();
			search(child, state.move(child.move));
			bool changed = updatePD(node);
		}while(!changed && node.phi != 0 && node.delta != 0);
	}
}
Node initnode(State state){
	Node node; node.move = state.lastmove();
	if(state.win()){       node.phi = 0;   node.delta = INF; }
	else if(state.loss()){ node.phi = INF; node.delta = 0;   }
	else{                  node.phi = 1;   node.delta = 1;   }
	return node;
}
bool updatePD(node){
	int phi = INF, delta = 0;
	foreach(node.children as child){
		phi = min(phi, child.delta);
		delta = delta + child.phi;
	}
	bool changed = (node.phi == phi && node.delta == delta);
	node.phi = phi; node.delta = delta;
	return changed;
}
\end{lstlisting}

\caption{Proof Number Search Pseudocode, shown in the negamax formulation, with the optimization to not propagate up if no changes occur}
\label{fig:pnscode}
\end{figure}

The pseudocode for Proof Number Search in the negamax formulation is shown in Figure \ref{fig:pnscode}.


\subsection{Transposition Table}

Proof number search uses an explicit tree which must be kept in memory, but the tree required is often bigger than available memory. One common approach to bounding the memory needed is to store the nodes in a transposition table instead of an explicit tree. This has the benefit of bounded memory as well as saving computation and memory on transpositions, at the cost of having to recompute nodes that are replaced in the transposition table. Even when a node needs to be recomputed, its children are often still in the transposition table, allowing for a quick recomputation. In many cases the transposition table can be several orders of magnitude smaller than would be needed to store the explicit tree.


\subsection{DF-PN: Depth First Proof Number search} \label{sec:DF-PN}

Depth First Proof Number search (DF-PN)\cite{nagai1999-DFPN} uses two thresholds which are set to express how long the MPN stays in the current subtree. The thresholds are calculated based on the realization that there is no need to update the parents $\phi$ and $\delta$ and do a new move selection if the next descent will go to the same child node. As long as the child's $\delta$ is smaller or equal to all of its siblings, an MPN still lies in its subtree. By staying within this subtree, fewer updates are needed, and locality is maintained, using fewer active nodes and needing less recomputation.

$n.th_\phi$ and $n.th_\delta$ are both set to $\infty$ at the root. $n.th_\phi$ and $n.th_\delta$ are computed as follows: $$n_c.th_\phi = n.th_\delta + n_c.\phi - \displaystyle\sum\limits_{i=0}^k n_i. \phi$$ $$n_c.th_\delta = min(n.th_\phi, n_2.\delta + 1)$$ where $n_c$ refers to the child with the smallest $\delta$ and $n_2$ is the child with the second smallest $\delta$. The search process continues at each node until $n.\phi \geq n.th_\phi$ or $n.\delta \geq n.th_\delta$.


\subsection{DF-PN+: DF-PN with Heuristics} \label{sec:DF-PN+}

So far proof number search assumes that all leaf nodes are equally hard to prove, using only proof and disproof numbers, or more generally, the shape of the tree. DF-PN+ adds two types of heuristic evaluation \cite{nagai-thesis}. 

$h(n)$ is a heuristic evaluation of the expected difficulty of proving or disproving a node and its subtree. DF-PN has a constant $h(n) = 1$. A small value means that this node is expected to be easy to (dis)prove while a big value means this node is expected to be hard to (dis)prove. This value is used at node initialization of non-terminal nodes: $$n.\phi = h_\phi(n), \quad n.\delta = h_\delta(n)$$

$cost(n, n_c)$ is a cost for moving from a node to a child, and affects the shape of the tree. DF-PN has a constant $cost(n, n_c) = 0$. Small values encourage narrow and deep trees while large values encourage wide and shallow trees. This value is a penalty to going deeper, and could be used to encourage deeper trees on moves that are evaluated to be better, or to find shallower solutions.

Adding a non-zero $cost$ function means changing the update formulas: $$ n.\phi = \displaystyle\min\limits_{i=0}^k (n_i.\delta + cost_\delta(n, n_c))$$ $$n.\delta = \displaystyle\sum\limits_{i=0}^k (n_i.\phi + cost_\phi(n, n_c))$$ as well as the threshold formulas:
$$n_c.th_\phi = n.th_\delta + n_c.\phi - \displaystyle\sum\limits_{i=0}^k (n_i.\phi - cost_\delta(n, n_i))$$ $$n_c.th_\delta = min(n.th_\phi, n_2.\delta + cost_\phi(n, n_2) + 1) - cost_\phi(n, n_c)$$

While the values initialized by $h$ are overwritten when its children are expanded, the $cost$ persists until the node is solved. Given fast and effective $h$ and $cost$ functions, DF-PN+ can solve problems significantly faster than DF-PN.

\subsection{The $1+\epsilon$ Trick} \label{sec:epstrick}

One weakness of Proof Number search in general is it requires a large amount of memory to store the tree. Depth first variants and transposition tables help, but many problems we want to solve need trees that are several orders of magnitude bigger than available memory. In cases where the memory limit is particularly tight, DF-PN can spend a huge amount of time regenerating subtrees. To reduce this effect, the $1+\epsilon$ Trick \cite{pawlewicz2007epsilon} increases the thresholds exponentially instead of linearly.

Consider the case where two children of the root (so $\infty$ thresholds) are both promising, but also hard to solve. Since Proof Number search likes keeping all children at similar values, the two will get roughly equal time. Once the subtrees become big, the time spent on one will overwrite the nodes from the other tree. Once its $\delta$ exceeds the other, the search will switch, taking up a large amount of memory, and overwriting the nodes from the subtree of the first child. Since the threshold is one higher than the second smallest, they will swap every time the tree gets rebuilt to exceed the other child by one. This linear increase in the threshold can mean an exponential increase in time.

If the threshold was not simply the second smallest + 1, but instead a multiple of it, the investment into this subtree wouldn't be overwritten so easily by its equally challenging siblings. The $1 + \epsilon$ trick therefore is to replace the $n_c.th_\delta$ threshold from DF-PN with $$n_c.th_\delta = min(n.th_\phi, n_2.\delta*(1 + \epsilon))$$ where $\epsilon$ is some small positive value. This way $n_c.th_\delta$ increases by a constant multiple instead of a constant. It leads to $n_c$ being called at most $log_{1+\epsilon}n.th = O(log n.th)$ times instead of linear times before reaching its parent's threshold.

This trick encourages DF-PN to be even more depth-first, causing lower memory usage, more efficient usage of the transposition table, and reducing the number of node recalculations. Note that it no longer follows the same order as the original PNS, sometimes expanding a node that isn't an MPN, and therefore it may do an unbounded amount of extra work.

$\epsilon = 0.25$ has been shown to work well in practice.


\section{Monte Carlo Tree Search}

For games where a very fast and strong evaluation function exists, alpha-beta is likely to be fast and strong, but no good heuristic is known for many games including Go and Havannah. Monte Carlo Tree Search (MCTS) is an algorithm for building and exploring a game tree that is based on statistics instead of a heuristic evaluation function. MCTS avoids using a heuristic by building its tree as guided by playing random games. While the random games have a very low playing strength, in aggregate random games will favour the player that is in a better position.

MCTS consists of four phases which together are called a simulation. Each simulation adds some experience to the tree, updating the expected chance of winning for the nodes it traverses. These winning rates are stored as the number of wins and the number of simulations through a node. The notation is: $n.v$ for the winning rate and $n.n$ for number of simulations. The four phases are:
\begin{description}
\item[Descent] This phase descends the game tree from the root to a leaf node N in the game tree. At each node one of the available moves is selected according to some criteria based on the current winning rate and possibly heuristic knowledge. When the Upper Confidence Bounds (UCB) formula is used, this is called Upper Confidence Bounds applied to Trees (UCT), but other formulas such as RAVE have been developed and are commonly used.
\item[Expansion] If the node N has experience from a previous simulation, its children are expanded, increasing the size of the tree.
\item[Rollout] A random game is played from N through the newly expanded children, to the end of the game. Heuristics can be used to make the moves less random and more representative of a real game. The strength of the overall algorithm is highly dependent on the average outcome of the random games being representative of the strength of the position.
\item[Back-propagation] The outcome of the random game in the rollout is back propagated to the moves chosen in the tree. The winning rate of the moves made by the player that won the rollout is increased while winning rate of the moves by the player that lost the rollout is decreased.
\end{description}

...diagram needed...

These four steps are repeated continually until a stopping condition is reached, such as running out of time or memory. At this point a move is chosen by some criteria. The three most common criteria are: most simulations, most wins, and highest lower confidence bound on winning rate. The most simulations is the most conservative, but if a counter-move was found late in the game, it may still be the most simulated even if it doesn't have the highest winning rate. The most wins is a little less conservative and will favour a late new-comer if it has almost caught up. The highest winning rate is quite risky since it may favour a move that has a very small subtree where a good counter move exists but hasn't been found yet. To deal with that a lower bound can be used, but a large confidence interval should be used to avoid choosing risky moves.

The pseudocode for MCTS is shown in Figure \ref{fig:mctscode}.

\begin{figure}

\begin{lstlisting}
Move mcts(State state){
	Node root = Node(state);
	while(!timeout)
		search(root, state);
	return root.bestchild();
}
int search(Node node, State state){ //0 = loss, 1 = win
	//rollout
	if(node.numchildren == 0 && node.sims == 0){
		while(!state.terminal())
			state.randmove();
		return (state.win() ? 1 : 0);
	}

	//expand
	if(node.numchildren == 0)
		foreach(state.successors as succ)
			node.addchild(Node(succ));

	//descent
	Node best;
	foreach(node.children as child)
		if(best.value() < child.value())
			best = child;

	int outcome = 1 - search(best, state.move(best.move));

	//back-propagate
	best.sims += 1;
	best.wins += outcome;
	return outcome;
}
\end{lstlisting}

\caption{Monte Carlo Tree Search Pseudocode, shown in the negamax formulation}
\label{fig:mctscode}
\end{figure}


Many extensions have been developed to increase the playing strength of MCTS. Some of these are explained below.

\subsection{UCT: Upper Confidence bounds as applied to Trees}

The most common and most famous formula in MCTS is UCT. It derives from Upper Confidence Bounds (UCB), which is famously used on the multi-armed bandit problem. It is used to balance exploitation and exploration when multiple options are available and each option returns a random distribution of reward. The amount of regret, ie the number of plays to non-optimal arms, should be minimized to maximize reward in the long term.

The UCT formula is: $$n_i.v + c*\sqrt{\frac{ln(n.n)}{n_i.n}}$$ where $c$ is a tunable constant to balance the exploration rate. This formula is used in the descent phase of MCTS to chose which move to make. Intuitively, moves with high winning rate should be exploited more, but moves with a small number of simulations as compared to the parent should be explored to improve the confidence. This formula is guaranteed to converge on the optimal move given infinite time and memory.




\subsection{RAVE: Rapid Action Value Estimate}

In basic MCTS many simulations are run per second, but the information about which moves were made during the rollouts is unused. A win or a loss is composed of many moves which contributed to that outcome, and often good moves during a rollout are also good moves if made earlier during the rollout or during the descent phase. This is a similar realization that lead to the History Heuristic. Thus, we can keep a winning rate for each move during the rollouts and use this to encourage exploration of moves that do well during rollouts. This information is gathered much quicker than pure experience, though it is less correlated to success, so should be phased out as real experience is gained. The notation used is $n.r$ for the rave winning rate and $n.m$ for the number of rave updates for a node.

Usually rave experience and real experience are combined as a linear combination, starting as only rave experience and asymptotically approaching only real experience:
$$ \beta*n_i.v + (1-\beta)*n_i.r $$
Several formulas for $\beta$ have been proposed. The simplest two formulas for $\beta$ are $$\beta = \frac{k}{k+n_i.n}$$ $$\beta = \sqrt{\frac{k}{k+3*n_i.n}}$$ both of which have a tunable constant $k$ which represents the midpoint, so how many simulations are needed for the rave experience and real experience to have equal weight.

David Silver computed an optimal formula for $\beta$ under the assumption of independence of estimates:
$$\beta = \frac{n_i.r}{n_i.n+n_i.r+4*n_i.n*n_i.r*b^2}$$
where $b$ is a tunable rave bias value.

In practice, RAVE leads to a large increase in playing strength for games such as Go and Havannah where the assumption that a good move is also good if played earlier.

\subsection{Heuristic Knowledge}\label{sec:heuristicknowledge}

While UCT is guaranteed to converge given infinite time, game specific knowledge can encourage it to find good moves faster. When a node is expanded, its children all start with no experience, so the default policy is to chose between them randomly. The simulation is going to be more representative of a good game, and lead to a better understanding of the minimax value if it chooses a good move first. Eventually the best move will receive the majority of the simulations, and we'll do better if this is true right from the beginning. Each game has its own heuristics, and Havannah specific ones are described in later chapters, but the way these heuristics is used is game independent.

The first way heuristic knowledge is used is to simply add fake experience to a node. Instead of initializing a node as $n_i.v = 0, n_i.n = 0$, good moves can be initialized with $n_i.v = c, n_i.v = c$, where c is a tunable constant, which effectively means that this node has some amount of wins attributed to it before any simulations have gone through it. This has the effect of allowing the node to look good for the first while even if it is unlucky, but the extra simulations will fade over time as the few extra wins becomes insignificant in the long run. Bad moves can similarly be initialized with fewer wins than simulations effectively depressing its early winning rate. Depending on your implementation, this may encourage the first few simulations to avoid the good moves though, due to their smaller confidence bounds compared to similar moves with the same high winning rate, which has the effect of making the grandparent move look bad. This knowledge could also be added as fake rave experience as well as or instead of to normal experience.

The other way heuristic knowledge is used is to add a knowledge term to the value formula. This leaves the the experience and confidence bounds alone, but gives a boost for the first few simulations to nodes with higher knowledge. This has the added benefit of being able to order the nodes by how large of a boost is given. The knowledge term should fall off with increasing experience. Two suggested knowledge terms are: $$\frac{n_i.k}{log(n_i.n)}, \quad \frac{n_i.k}{\sqrt{n_i.n}}$$ where $n_i.k$ is the knowledge value for the node $n_i$.


\subsection{Rollout Policy}

As mentioned above, the strength of MCTS is highly dependent on the average outcome of the rollouts being representative of the strength of the position. When a player who is in a good position has an easy defence to a devastating attack, but fails to defend, the outcome is not representative of the strength of the original position. Decreasing the randomness by enforcing defences against devastating attacks can bias the outcome, but usually leads to higher quality and more representative games, leading to a stronger player. Most rollout policies used in real programs are game specific, but a few game independent ones are mentioned here.

Instead of pure random, a weighted random scheme can be used. Moves that have good experience in the tree can be selected at a higher probability to poor moves. This could be based on real experience, rave experience or heuristic knowledge as described in the section \ref{sec:heuristicknowledge}.

The Last Good Reply scheme can be used, where the moves made by the player that won a rollout are saved for use in later rollouts when similar situations occur. When these moves fail to lead to a win in a later rollout, they may be removed from the list of replies.

All possible moves can be checked to see if they lead to an instant win, in which case that move should be made. Similarly if this turn is skipped would each move lead to an instant loss if not made, in which case the defensive should be forced.



\subsection{Other extensions}

should any of these be described?

\begin{itemize}
\item parallelization
\item dynamic widening
\item Solution backups
\item multiple rollouts per simulation
\item avoid symmetries or transpositions
\item first player urgency
\end{itemize}














  % Rules and properties of Havannah
  \chapter[Havannah]{\label{havannah} \LARGE Rules and Properties of Havannah}
  % vim: set wrap

\section{Rules of Havannah}


\begin{figure}
\centering
\begin{HavannahBoard}[board size=6,coordinate style=classical]
\HStoneGroup[color=black,label=$\mathcal F$]{e10,f10,g10,g9,h9,h8,i8,j8,h7,h6,h5,i5,i4,k8}
\HStoneGroup[color=white,label=$\mathcal B$]{a1,a2,b3,c3,d4,e4,e3,e2,f2,f1}
\HStoneGroup[color=gray,label=$\mathcal R$]{e7,e8,d8,c8,b7,b6,b5,c5,d6}
\end{HavannahBoard}
\caption{The three winning conditions as shown on a size 6 Havannah board}
\label{fig:rules}
\end{figure}

Havannah is a connection game invented in 1979 by Christian Freeling. It is a two player, zero-sum, perfect information game played on a hexagonal board. Each turn a player places a piece on the board in alternating play. Pieces are never moved nor removed from the board after their initial placement. The three winning conditions are shown in Figure \ref{fig:rules}:
\begin{itemize}
	\setlength{\itemsep}{0pt}
	\setlength{\parskip}{0pt}
	\setlength{\parsep}{0pt}
	\item A \textbf{Bridge} is a group of stones that links 2 corners, for example the stones labelled $\mathcal B$ in Figure \ref{fig:rules}
	\item A \textbf{Fork} is a group of stones that links 3 edges (corners are not part of either edge), for example the stones labelled $\mathcal F$ in Figure \ref{fig:rules}
	\item A \textbf{Ring} is a group of stones that surround at least one cell (which can be empty or filled by either player), for example the stones labelled $\mathcal R$ in Figure \ref{fig:rules}
\end{itemize}

Havannah is played on board sizes ranging from 4 to 10 cells per side. Stronger players prefer bigger boards due to the larger component of strategy compared to the small boards where tactics dominate. In 2002, Christian Freeling offered \euro 1000 for any program that beats him in just one in ten games on size 10 by 2012.

Havannah is played by a few thousand players around the world, primarily on Little Golem\footnote{http://littlegolem.net} and similar sites. It is also played by computer programs at the International Computer Games Association games tournaments yearly.



\section{Coordinate System}

Several coordinate systems exist, but the one that will be used here is the one used in HavannahGui, in the Little Golem SGF files, and that has some nice mathematical properties. An example board is shown in Figure \ref{fig:coordinates} with each cell marked with its coordinate location. Next to it is a representation of a board as shown on a square grid. The empty points in the square grid are unused for the purposes of this representation. This square representation is often used to represent the board in memory. The size of the board is the number of cells along one short edge, or the radius of the board, not the diameter.

\begin{figure}
\centering

	\subfloat[]{
		\begin{HavannahBoard}[board size=3,coordinate style=classical,hex height=22pt]
		\HStoneGroup[color=white,label=a1]{a1}
		\HStoneGroup[color=white,label=a2]{a2}
		\HStoneGroup[color=white,label=a3]{a3}
		\HStoneGroup[color=white,label=b1]{b1}
		\HStoneGroup[color=white,label=b2]{b2}
		\HStoneGroup[color=white,label=b3]{b3}
		\HStoneGroup[color=white,label=b4]{b4}
		\HStoneGroup[color=white,label=c1]{c1}
		\HStoneGroup[color=white,label=c2]{c2}
		\HStoneGroup[color=white,label=c3]{c3}
		\HStoneGroup[color=white,label=c4]{c4}
		\HStoneGroup[color=white,label=c5]{c5}
		\HStoneGroup[color=white,label=d2]{d2}
		\HStoneGroup[color=white,label=d3]{d3}
		\HStoneGroup[color=white,label=d4]{d4}
		\HStoneGroup[color=white,label=d5]{d5}
		\HStoneGroup[color=white,label=e3]{e3}
		\HStoneGroup[color=white,label=e4]{e4}
		\HStoneGroup[color=white,label=e5]{e5}
		\end{HavannahBoard}
	}
	\subfloat[]{
		\raisebox{58pt}{
		\begin{tabular}{c|ccccc}
		  &  1 &  2 &  3 &  4 &  5 \\ \hline
		a & a1 & a2 & a3 &    &    \\
		b & b1 & b2 & b3 & b4 &    \\
		c & c1 & c2 & c3 & c4 & c5 \\
		d &    & d2 & d3 & d4 & d5 \\
		e &    &    & e3 & e4 & e5 \\
		\end{tabular}
		}
	}
\caption{(a) The coordinates as drawn on a size 3 board. (b) The same board as represented on a square grid.}
\label{fig:coordinates}
\end{figure}

This coordinate system can be defined mathematically as the integer points on the cube (x,y,z) in \{1-size, ... , size-1\}$^3$ where $x + y + z = 0$. This is a plane right through the cube that forms a hexagon of points. Given this representation, the distance between any two points can be calculated with
$$d = (|x_1-x_2| + |y_1-y_2| + |z_1-z_2|)/2$$
and the edge and corner cells all have the property that
$$(|x| + |y| + |z|)/2 = s$$
where $s$ is the board size. Further mathematical analysis is available at \url{http://www.iwriteiam.nl/Havannah.html}, but is unneeded for the rest of this thesis.



\section{Complexity}

A very naive calculation of the state space complexity of Havannah is simply $T_N = 3^N$ where N is the number of cells on the board. This includes many states that are unreachable purely based on the players having an uneven number of moves, such as all cells being played by player 1. A much more accurate calculation is the sum of states where both players have made equal number of moves plus the sum of all states where player 1 has  made on more move. This can be expressed as the formula:
$$T_N = \sum_{i = 0}^{(N-1)/2} [ {N \choose i}*{N-i \choose i} + {N \choose i + 1}*{N - i - 1 \choose i}]$$
Dividing these two gives an error of approximately $\sqrt{N}$, so a fairly close approximation of $T_N = 3^N/\sqrt{N}$. These do not take symmetry or rotations into account, which gives approximately a 12 fold reduction in states. It also doesn't take the rules of the game into account, so includes positions where both players have winning formations or one player has multiple winning formations.

The state space complexity of Havannah is shown in Figure \ref{table:complexity}, with various other board games listed for comparison.

\begin{table}
  \centering
\begin{tabular}{lrr|lr}
Havannah & Cells & States       & Game      & States \\ \hline
Size 4   &    37 & $6*10^{15}$  & Connect 4 & $10^{14}$ \\
Size 5   &    61 & $1*10^{27}$  & Checkers  & $10^{20}$ \\
Size 6   &    91 & $2*10^{41}$  & Hex 8x8   & $10^{29}$ \\
Size 7   &   127 & $3*10^{58}$  & Hex 11x11 & $10^{56}$ \\
Size 8   &   169 & $3*10^{78}$  & Chess     & $10^{46}$ \\
Size 9   &   217 & $2*10^{101}$ & Go 9x9    & $10^{38}$ \\
Size 10  &   271 & $1*10^{127}$ & Go 19x19  & $10^{171}$\\

\end{tabular}
	\caption{State complexity of Havannah. Other board games are shown for comparison.}
	\label{table:complexity}
\end{table}

\section{Properties of Havannah}

Havannah is considered hard to play for computers for several reasons including the lack of a good heuristic evaluation function, few expert games, and a large state space complexity. It is often compared to Hex, which is also a connection game, but very few of the mathematical properties of  Hex apply in Havannah. In this section I explain some of the properties of Havannah, especially in contrast to the better known properties of Hex.

\subsection{Dead Cells}

Strong hex programs reduce the moves under consideration by avoiding playing in dead cells. Dead cells are cells that provably cannot affect the outcome of the game. The five basic Hex dead cells are shown in Figure \ref{fig:hexdeadcells}. They are dead because any chain of stones that passes through them already has a path through the existing stones.

\begin{figure}
  \centering
\begin{tabular}{ccccc}
\begin{HavannahBoard}[board size=2,coordinate style=classical,show coordinates=false]
\HStoneGroup[color=white]{a1,b1,c2,c3}
\end{HavannahBoard}
&
\begin{HavannahBoard}[board size=2,coordinate style=classical,show coordinates=false]
\HStoneGroup[color=black]{b3}
\HStoneGroup[color=white]{a1,b1,c2}
\end{HavannahBoard}
&
\begin{HavannahBoard}[board size=2,coordinate style=classical,show coordinates=false]
\HStoneGroup[color=black]{a2,b3}
\HStoneGroup[color=white]{b1,c2}
\end{HavannahBoard}
&
\begin{HavannahBoard}[board size=2,coordinate style=classical,show coordinates=false]
\HStoneGroup[color=black]{a1,a2,b3}
\HStoneGroup[color=white]{c2}
\end{HavannahBoard}
&
\begin{HavannahBoard}[board size=2,coordinate style=classical,show coordinates=false]
\HStoneGroup[color=black]{a1,a2,b3,c3}
\end{HavannahBoard}

\end{tabular}
	\caption{Hex dead cell patterns: The center of each pattern cannot help either player form a winning connection in Hex.}
	\label{fig:hexdeadcells}
\end{figure}

Unfortunately these cells can have an effect on the outcome of Havannah. While they cannot affect a fork or a bridge connection, they can still be part of a winning ring which surrounds one of the existing stones as illustrated in Figure \ref{fig:ringdeadcells}. This idea can be used against all 5 of the simple hex dead cell patterns in numerous ways.


\begin{figure}
  \centering
\begin{tabular}{ccc}
\begin{HavannahBoard}[board size=3,coordinate style=classical,show coordinates=false]
\HStoneGroup[color=white]{b2,c2,d3,d4}
\end{HavannahBoard}
&
\begin{HavannahBoard}[board size=3,coordinate style=classical,show coordinates=false]
\HStoneGroup[color=white]{b2,c2,d3,d4}
\HStoneGroup[color=light gray,label=1]{b1}
\HStoneGroup[color=light gray,label=2]{c1}
\HStoneGroup[color=light gray,label=3]{d2}
\HStoneGroup[color=light gray,label=4]{c3}
\end{HavannahBoard}
&
\begin{HavannahBoard}[board size=3,coordinate style=classical,show coordinates=false]
\HStoneGroup[color=white]{b1,b2,c1,c3,d2,d3}
\HStoneGroup[color=light gray]{c2,d4}
\end{HavannahBoard}
\end{tabular}
	\caption{Starting with the hex dead cell pattern on the left, and adding stones 1-4 leads to the ring on the right which intersects the dead cell pattern}
	\label{fig:ringdeadcells}
\end{figure}

Note that if the ring is made bigger than a simple 6-ring, any ring that uses the dead cell would also work going around it, so to create Havannah dead cell patterns, we simply add a stone of the opposing colour next to each existing stone to block the encircling 6-ring. A few examples are shown in Figure \ref{fig:havdeadcells}. Unfortunately these patterns are so rare that they aren't worth looking for.

\begin{figure}
  \centering
\begin{tabular}{ccccc}

\begin{HavannahBoard}[board size=3,coordinate style=classical,show coordinates=false,hex height=14pt]
\HStoneGroup[color=white]{b2,c2,d3,d4}
\HStoneGroup[color=black]{b1,e4}
\end{HavannahBoard}
&
\begin{HavannahBoard}[board size=3,coordinate style=classical,show coordinates=false,hex height=14pt]
\HStoneGroup[color=white]{b2,c2,d3, b4}
\HStoneGroup[color=black]{c4, b1,e3}
\end{HavannahBoard}
&
\begin{HavannahBoard}[board size=3,coordinate style=classical,show coordinates=false,hex height=14pt]
\HStoneGroup[color=white]{c2,d3, b4}
\HStoneGroup[color=black]{b3,c4, d2}
\end{HavannahBoard}
&
\begin{HavannahBoard}[board size=3,coordinate style=classical,show coordinates=false,hex height=14pt]
\HStoneGroup[color=white]{d3, a2,b4}
\HStoneGroup[color=black]{b2,b3,c4, e3}
\end{HavannahBoard}
&
\begin{HavannahBoard}[board size=3,coordinate style=classical,show coordinates=false,hex height=14pt]
\HStoneGroup[color=black]{b2,b3,c4,d4}
\HStoneGroup[color=white]{b1,a3,d5}
\end{HavannahBoard}

\end{tabular}
	\caption{A few examples of Havannah dead cell patterns}
	\label{fig:havdeadcells}
\end{figure}



\subsection{Virtual Connections}

A virtual connection is a connection between two stones or a stone and an edge that can be completed even if the opponent makes the first move. The simplest virtual connection is shown in Figure \ref{fig:simplevc}. The two black stones are virtually connected, since if white plays in one of the cells, black can complete the connection by playing in the other. In Hex, virtual connections are guaranteed since there is no reason to not complete the connection, but this is not true in Havannah. In Figure \ref{fig:breakvc}, black has a virtual connection between his two groups, but white can force a defence against a ring threat as shown in Figure \ref{fig:brokenvc}, which allows white to sever the black virtual connection. These threats are rare in practice, but do occur and can't be ignored.


\begin{figure}
  \centering

	\subfloat[]{\label{fig:simplevc}
		\begin{HexBoard}[board size=2,show coordinates=false,show hexes=true]
		\HHexGroup {a1,b1,a2,b2}
		\draw [thick,dotted] (a1)..controls(a2)..(b2);
		\draw [thick,dotted] (a1)..controls(b1)..(b2);
		\HStoneGroup[color=black]{a1,b2}
		\end{HexBoard}
	}
	\subfloat[]{\label{fig:breakvc}
		\begin{HavannahBoard}[board size=3,coordinate style=classical,show coordinates=false]
		\HStoneGroup[color=black]{d2,d3,c4,b4}
		\HStoneGroup[color=white]{b1,c2,a2,b3}
		\end{HavannahBoard}
	}
	\subfloat[]{\label{fig:brokenvc}
		\begin{HavannahBoard}[board size=3,coordinate style=classical,show coordinates=false]
		\HStoneGroup[color=black]{d2,d3,c4,b4}
		\HStoneGroup[color=white]{b1,c2,a2,b3}
		\HGame{c3,a1,d4}
		\end{HavannahBoard}
	}

	\caption{(a) The simplest virtual connection. (b) Virtual connections are not guaranteed. (c) A threat can force a response other than maintaining the connection.}
	\label{fig:ringvc}
\end{figure}



\subsection{Draws}

Unlike Hex where a filled board must have a winner, draws are possible in Havannah. Figure \ref{fig:drawfilled} shows the filled board of the game that ended in a draw. Figure \ref{fig:drawnowin} is the board once no wins are possible even if one of the players were to pass all their remaining moves. Effectively all remaining cells are dead cells. Figure \ref{fig:drawproven} is the same board after move 20, where the game is already a proven draw. Draws are possible on all board sizes above 3. They occur occasionally on size 4 and but are very rare above on size 5 and above.

\begin{figure}
	\centering
	\subfloat[]{\label{fig:drawfilled}
		\begin{HavannahBoard}[board size=4,coordinate style=classical,show coordinates=false]
		\HGame{g7,a1,f5,g4,e3,d2,e2,d1,c2,d3,c3,f7,d6,b4,a4,b5,a3,e5,c4,b3, a2,c1,b1,f6,e4,d4,g6,e7,d7,b2,c5, f3,f4,g5,d5,e6,c6}
		\end{HavannahBoard}
	}
	\subfloat[]{\label{fig:drawnowin}
		\begin{HavannahBoard}[board size=4,coordinate style=classical,show coordinates=false]
		\HGame{g7,a1,f5,g4,e3,d2,e2,d1,c2,d3,c3,f7,d6,b4,a4,b5,a3,e5,c4,b3, a2,c1,b1,f6,e4,d4,g6,e7,d7,b2,c5}
		\end{HavannahBoard}
	}
	\subfloat[]{\label{fig:drawproven}
		\begin{HavannahBoard}[board size=4,coordinate style=classical,show coordinates=false]
		\HGame{g7,a1,f5,g4,e3,d2,e2,d1,c2,d3,c3,f7,d6,b4,a4,b5,a3,e5,c4,b3}%, a2,c1,b1,f6,e4,d4,g6,e7,d7,b2,c5}
		\end{HavannahBoard}
	}
	\caption{(a) A filled board ending as a draw. (b) After move 30 no wins are possible. (c) Proven draw after move 20 if both players maintain their virtual connections.}
	\label{fig:draw}
\end{figure}



\subsection{Simultaneous forced wins: Race to win}

In Hex, winning formations are mutually exclusive, so if one player has a forced win through virtual connections, he is guaranteed to win. In Havannah winning formations are not mutually exclusive. Figure \ref{fig:rules} shows 3 completed winning formations at the same time, though this is obviously not  valid board configuration. While virtual connections can be broken, the formations needed to do so are often not present, in which case the virtual connections are guaranteed. A series of virtually connected stones or chains of stones that can lead to a guaranteed victory are called a Frame. Figure \ref{fig:racea} shows a situation where both players have a frame of length 2. This means they can force a win in 2 moves, so the first player to make a move wins. Figure \ref{fig:raceb} shows a situation where both players have a forced win in 3 moves, with black to move. One of white's moves threatens a faster ring victory, which is easily blocked, but which gives white the move advantage and the win.


\begin{figure}
	\centering
	\subfloat[]{\label{fig:racea}
		\begin{HavannahBoard}[board size=4,coordinate style=classical,show coordinates=false]
		\HStoneGroup[color=white]{d1,e3,g4}
		\HStoneGroup[color=black]{a4,c5,d7}
		\end{HavannahBoard}
	}
	\subfloat[]{\label{fig:raceb}
		\begin{HavannahBoard}[board size=4,coordinate style=classical,show coordinates=false]
		\HStoneGroup[color=white]{b5,c5,c4,b3,c2}
		\HStoneGroup[color=black]{f6,f5,e3,f7}
		\end{HavannahBoard}
	}
	\subfloat[]{\label{fig:racec}
		\begin{HavannahBoard}[board size=4,coordinate style=classical,show coordinates=false]
		\HStoneGroup[color=white]{b5,c5,c4,b3,c2}
		\HStoneGroup[color=black]{f6,f5,e3,f7}
		\HGame[first player=black]{g6,a3,a4,b2,f4,c1}
		\end{HavannahBoard}
	}
	\caption{(a) Both players have a forced win in 2 moves, the first player to move wins. (b) Both players have a forced win in 3 moves, black to move. (c) White gains a free move with a ring threat to win.}
	\label{fig:race}
\end{figure}



\subsection{Multiple and Complex Win Conditions}

The rules of Havannah are quite simple, and the three win conditions are easy to describe, but they aren't so simple to reason about.

Both bridges and forks can be found in near O(1) with the union find algorithm and some bit counting. Each stone placed on a corner or edge has a bit associated with that corner or edge set, and as stones form chains, the leader of the group ORs together the bits for all the corners and edges the group is connected to. Once the group reaches two corners or three edges, it's a win.

For a simple implementation, this works well, but humans think in terms of frames and spend a fair amount of time and effort building frames and counting how long a frame will take to complete. If the goal is to understand how long a frame will take to complete, the possible frames must be enumerated. In that case, there are ${6 \choose 2} = 15$ possible bridge wins and ${6 \choose 3} = 20$ possible fork wins, which must be considered independently.

There are many more ways for rings and ring frames to occur, and rings are much tougher to detect. One obvious way to detect rings is to flood fill the board and look for unreachable areas. This method is not an incremental algorithm, so ignores the assertion from previous moves that there are no rings, and is very slow. There are two incremental approaches, both of which are fast.

The first is to do a depth-first search starting from the most recently placed stone, as shown in Figure \ref{fig:dfring}. The search is only started if the group is at least 6 stones, and if the last stone joins one group of stones twice. From the starting stone, it searches in 4 adjacent directions (4 is enough because any ring must start from one of the four directions even if it cycles back through the other two), continuing only in the forward direction to the next 3 stones. By avoiding any sharp turns, the minimum cycle is 6 and any path back to the starting stone is a ring.

\begin{figure}
	\centering
	\subfloat[]{\label{fig:dfringa}
		\begin{HavannahBoard}[board size=3,coordinate style=classical,show coordinates=false]
		\HStoneGroup[color=light gray]{b2}
		\HStoneGroup[color=white]{c3,c4,b4,a3,a2, d3,d4}
		\draw [thick,->] (b2) -- (a1);
		\draw [thick,->] (b2) -- (b1);
		\draw [thick,->] (b2) -- (c2);
		\draw [thick,->] (b2) -- (c3);
		\end{HavannahBoard}
	}
	\subfloat[]{\label{fig:dfringb}
		\begin{HavannahBoard}[board size=3,coordinate style=classical,show coordinates=false]
		\HStoneGroup[color=light gray]{b2}
		\HStoneGroup[color=white]{c3,c4,b4,a3,a2, d3,d4}
		\draw [thick]    (b2) -- (c3);
		\draw [thick,->] (c3) -- (d3);
		\draw [thick,->] (c3) -- (d4);
		\draw [thick,->] (c3) -- (c4);
		\end{HavannahBoard}
	}
	\subfloat[]{\label{fig:dfringc}
		\begin{HavannahBoard}[board size=3,coordinate style=classical,show coordinates=false]
		\HStoneGroup[color=light gray]{b2}
		\HStoneGroup[color=white]{c3,c4,b4,a3,a2, d3,d4}
		\draw [thick]    (b2) -- (c3);
		\draw [thick] (c3) -- (d3);
		\draw [thick] (c3) -- (d4);
		\draw [thick] (c3) -- (c4);
		\draw [thick,->] (d3) -- (d2);
		\draw [thick,->] (d3) -- (e3);
		\draw [thick,->] (d3) -- (e4);
		\draw [thick,->] (d4) -- (e4);
		\draw [thick,->] (d4) -- (e5);
		\draw [thick,->] (d4) -- (d5);
		\draw [thick,->] (c4) -- (d5);
		\draw [thick,->] (c4) -- (c5);
		\draw [thick,->] (c4) -- (b4);
		\end{HavannahBoard}
	}
	\subfloat[]{\label{fig:dfringd}
		\begin{HavannahBoard}[board size=3,coordinate style=classical,show coordinates=false]
		\HStoneGroup[color=light gray]{b2}
		\HStoneGroup[color=white]{c3,c4,b4,a3,a2, d3,d4}
		\draw [thick]    (b2) -- (c3);
		\draw [thick]    (c3) -- (c4);
		\draw [thick]    (c4) -- (b4);
		\draw [thick,->] (b4) -- (a3);
		\draw [thick,->] (a3) -- (a2);
		\draw [thick,->] (a2) -- (b2);
		\end{HavannahBoard}
	}
	\caption{Depth-first ring detection. Gray stone is the most recently placed and the start of the search. (a) Search after 1 step. (b) Search after 2 steps. (c) Search after 3 steps. (d) Search after 6 steps, ring found.}
	\label{fig:dfring}
\end{figure}


Ring detection can be also be done in O(1) time as shown in Figure \ref{fig:o1ring}. Rings occur when the most recently placed stone touches the same group twice with them being separated by empty space or a different group on either side. The only circumstance where that isn't true is a filled ring, which can only happen in a small number of ways and is easy to detect. Figure \ref{fig:o1ringa} shows the common case where a stone joins a group twice and has empty space in the middle of the ring and on the opposite side. If the center of the ring is filled, as is the case in Figure \ref{fig:o1ringb}, a check of the neighbours is only enough to deduce that it may be a ring. A small search for a possible 6-ring is enough to conclude whether it is a ring or not. The reason no bigger search is needed is because any bigger ring would have been found earlier by the other criteria. The worst case, shown in Figure \ref{fig:o1ringd} is when the stone has 5 adjacent neighbours, in which case 4 searches for 6-rings are needed.

\begin{figure}
	\centering
	\subfloat[]{\label{fig:o1ringa}
		\begin{HavannahBoard}[board size=3,coordinate style=classical,show coordinates=false]
		\HStoneGroup[color=light gray]{b2}
		\HStoneGroup[color=white]{c3,c4,b4,a3,a2, d3,d4}
		\draw [thick,->] (b2) -- (a2);
		\draw [thick,->] (b2) -- (c3);
		\end{HavannahBoard}
	}
	\subfloat[]{\label{fig:o1ringb}
		\begin{HavannahBoard}[board size=3,coordinate style=classical,show coordinates=false]
		\HStoneGroup[color=light gray]{b2}
		\HStoneGroup[color=white]{c3,c4,b4,a3,a2, d3,d4,b3}
		\draw [thick,->] (b2) -- (c3);
		\draw [thick,->] (b2) -- (a2);
		\draw [thick,->] (b2) -- (b3);
		\end{HavannahBoard}
	}
	\subfloat[]{\label{fig:o1ringc}
		\begin{HavannahBoard}[board size=3,coordinate style=classical,show coordinates=false]
		\HStoneGroup[color=light gray]{b2}
		\HStoneGroup[color=white]{c3,c4,b4,a3,a2, d3,d4,b3}
		\draw [thick]    (b2) -- (c3) -- (c4) -- (b4) -- (a3) -- (a2) -- (b2);
		\end{HavannahBoard}
	}
	\subfloat[]{\label{fig:o1ringd}
		\begin{HavannahBoard}[board size=3,coordinate style=classical,show coordinates=false]
		\HStoneGroup[color=light gray]{c3}
		\HStoneGroup[color=white]{d3,d4,c4,b3,b2, d2,e4,e5,c5,a3,a2,b1}
		\draw [thick,->] (c3) -- (d3);
		\draw [thick,->] (c3) -- (d4);
		\draw [thick,->] (c3) -- (c4);
		\draw [thick,->] (c3) -- (b3);
		\draw [thick,->] (c3) -- (b2);
		\draw [dotted]    (c3) -- (c4) -- (b4) -- (a3) -- (a2) -- (b2) -- (c3);
		\draw [dotted]    (c3) -- (d4) -- (d5) -- (c5) -- (b4) -- (b3) -- (c3);
		\draw [dotted]    (c3) -- (d3) -- (e4) -- (e5) -- (d5) -- (c4) -- (c3);
		\draw [dotted]    (c3) -- (b2) -- (b1) -- (c1) -- (d2) -- (d3) -- (c3);

		\end{HavannahBoard}
	}
	\caption{O(1) ring detection. Gray stone is the most recently placed. (a) Stone joins the white group twice with an empty stone between the two white stones, obviously a ring enclosing the empty stone. (b) Stone joins the white group three times, no empty stone, leads to (c) a tiny search. (d) Worst case has 5 neighbours, does 4 tiny searches.}
	\label{fig:o1ring}
\end{figure}


  
  % The first real chapter
  % This one has some footnotes
  % You MUST reference any publications that contain work you performed in this manner
  % Additionally it is nice to acknowledge if some result you are presenting was obtained by someone else
  % Feel free to change the directory naming schema - I used representitive topic names for both the folder and the chapter .tex
  % The paper reference includes a hyperlink (\href) to the paper, but it is not necessary.
  \chapter[Playing Havannah]{\label{playing} \LARGE Playing Havannah }
  % vim: set wrap



\begin{itemize}

\item decreasing rave
\item multiple rollouts
\item move choice: most wins, not most sims nor highest lower bound
\item use exploration/rave some fraction of the time
\item dynamic widening?
\item keep tree between moves
\item pondering
\item parallelization, scaling, virtual loss
\item proofs
	\begin{itemize}
		\item extra ply or two at node expansion with null move
		\item macro moves for forced moves
	\end{itemize}
\item Havannah specific stuff
	\begin{itemize}
		\item vc, rollout/knowledge
		\item connectivity
		\item group size
		\item local reply
		\item locality
		\item distance to win
		\item check for rings only sometimes
	\end{itemize}
\end{itemize}


  
  % A second real chapter, this time without footnotes
  \chapter[Solving Size 4 Havannah!]{\label{bchapterlabel} \LARGE Solving Size 4 Havannah with MCTS}
  \begin{enumerate}
\item Proof Number Search

\item DFPN

\item MCTS

\end{enumerate}


  % Final chapter, conclusions
  \chapter[Conclusions]{\label{conc} \LARGE Conclusions}
  % vim: set wrap


  
  % Appendices
  % Comment out if you don't plan on having any appendices
%  \appendix
%  \chapter[Statistical Derivations]
%  {\label{appendstats} \LARGE Appendix: Statistical Derivations}
%  \input{./appendices/appendices}  
  
  % Ta-da!  A thesis
  % ********************************************************************************
  
  % References - with commands to modify size and spacing
  \phantomsection
  \addcontentsline{toc}{chapter}{References}
  \small
  \renewcommand{\baselinestretch}{0.25}
  \bibliography{thesis}

\end{document}
